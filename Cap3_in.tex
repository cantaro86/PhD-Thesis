

\chapter{Multinomial methods for Variance Gamma}\label{Chapter3}
%\blindtext
\minitoc% Creating an actual minitoc

\vspace{5em}


 
 
In this thesis we decided to work with the Variance Gamma (VG) process because it has nice analytical properties and 
it reproduces quite well the statistical features of the stock dynamics (see for instance \cite{Cont} and \cite{Ait12}).

To support this statement, 
we present in Figure \ref{FigPDF} some examples of histograms of
daily log-returns of the four indices:  
the \textbf{S\&P 500} Stock Index, 
the \textbf{KOSPI} (Korea Composite Stock Price Index), 
\textbf{XAO} (All Ordinaries Australian Index)  
and \textbf{TAIEX} (Taiwan Capitalization weighted Stock Index).
In the pictures we show the fit of the Normal and Variance Gamma (VG) densities, using the market data. 
It is clear that the 
VG density reproduces much better the high peaks near the origin and the heavy tails of the empirical distribution.

\begin{figure}[t!]
 \centering
 \includegraphics[width=0.47\textwidth]{sp500.png}
 ~
 \includegraphics[width=0.47\textwidth]{twii.png}
 ~
 \includegraphics[width=0.47\textwidth]{kospi.png}
 ~
 \includegraphics[width=0.47\textwidth]{xao.png}
 % VG4_pdf.png: 0x0 pixel, 0dpi, 0.00x0.00 cm, bb=
 \caption{Histograms of daily log-returns for S\&P500, KOSPI, XAO and TAIEX, from 1 January 1988 to 9 December 2016. Source: Yahoo Finance.
 The dashed line corresponds to the VG density (\ref{VG_density}). 
 The continuous line is the normal density. 
 The parameters are obtained by the \emph{method of moments}. For more details on the parameter estimation for the VG density we refer to \cite{Se04}.}
 \label{FigPDF}
\end{figure} 

The VG process was first presented in the context of option pricing 
in \cite{MaMi91}, where it has been used for pricing European options.
European vanilla options can be easily priced by the analytical formula presented in \cite{MCC98} and exotic options
can be priced numerically by several common techniques.
Monte Carlo methods for VG are presented in \cite{Fu00}. 
A finite difference scheme for the VG Partial Integro-Differential Equation (PIDE) 
is described in \cite{CoVo05b}. In \cite{CaMa98}, the authors show how to price options by a Fourier transform approach.
The problem for American options is considered in \cite{Al05}, \cite{Oo05} and \cite{HiMa01}, where the authors present different finite difference 
schemes to solve the American VG PIDE.

The tree method was first introduced by \cite{CRR79} for a market where the log-price can change only in two different ways: 
an upward jump, or a downward jump. For this reason the model is called \emph{binomial model}. The authors 
prove that when the number of time steps goes to infinity, the discrete random walk of the log-price converges to the Brownian motion
and the option price converges to the Black-Scholes price.
The \emph{multinomial model} is a generalization of the binomial model, and at each time step it considers 
more than just two possible future states.
A general multinomial method for pricing European and American options under exponential L\'evy processes is described in \cite{MaSoSz}. 
In \cite{KeWe06} the authors consider a multinomial method for general exponential L\'evy processes based on the moment matching condition.
Other methods based on the moment matching condition are for instance \cite{HaMac10}, with applications to the Normal Inverse Gaussian process, and 
\cite{See13} with applications to the VG process.  
In the present work we consider a multinomial discretization based on the cumulant matching condition as explained in \cite{YaPr01}, \cite{YaPr03} and \cite{YaPr06}. 


In Section \ref{sec3_ch3} we review the construction of the multinomial tree, following the method of
moment matching proposed in \cite{YaPr01}. We prove that the multinomial tree converges to the continuous time jump process that we have
introduced to approximate the VG process.
In Section \ref{sec4_ch3}, we describe the algorithm for pricing options with the 
multinomial method and show the numerical results for European and American options.




\section{The multinomial method} \label{sec3_ch3}
In this section we introduce the multinomial method proposed in \cite{YaPr06}. 
The stock price is represented by a Markov chain with $L$ possible future states at each time. 
In this setting, the time $t \in [t_0,T]$ is discretized as $t_n = t_0 + n\Delta t$ for $n=0, ... ,N$ and 
$\Delta t = (T-t_0)/N$. We denote the stock price at time $t_n$ as $S_{t_n} = S_n$.

Let us consider the up/down factors $u>d>0$, and write the discrete evolution of the stock price $S_n$ as:
\begin{equation}\label{Discr_S}
  S_{n+1} = u^{L-l}d^{l-1} S_n  \hspace{3em} l=1, ... , L  
\end{equation}
where each future state has transition probability $p_l$, satisfying $\sum_{l=1}^L p_l = 1$.
The value of the stock at time $t_n$ can assume $j \in [1,...,n(L-1)+1]$ possible values:
\begin{equation}\label{Discr_S2}
  S_{n}^{(j)} = u^{n(L-l)+1-j}d^{j-1} S_0.  
\end{equation}
The multinomial tree is recombining if $u/d = c$, for $c>1$.
In the present work we only consider five branches, $L=5$. As we will see in the next sections this number of branches is 
enough to model the features of a stochastic process up to its fourth moment.


\subsection{Moment matching}

In order to determine the parameters of the Markov chain we require that its local moments are equal to that of the continuous process.
Let us consider a VG process with drift $(r-w)$, where $w$ is the martingale correction defined in (\ref{parameter_w}): 
%and derive the continuous process for the log-variable $Y_t = \log(S_t)$: 
\begin{align}\label{log_martingale}
 Y_{t+\Delta t}-Y_t &= (r-w)\Delta t + \int_{\R} x N(\Delta t,dx) \\ \nonumber
		    &= (r-w + \theta)\Delta t + \int_{\R} x \tilde N(\Delta t,dx)
\end{align}
The parameter $\theta = \int_{\R} x \nu(dx) = \E \bigl[ \int_{\R} x N(1,dx) \bigr]$ is the expected value of the VG process in (\ref{VG_process}), 
when $\Delta t=1$. 
The integral with respect to the compensated Poisson measure $\tilde N(\Delta t,dx)$ is a martingale as discussed in Section \ref{random_measures}.

We can pass to log-prices $Y_n = \log(S_n)$ in the discrete Eq. (\ref{Discr_S}), and write it as the sum of a drift component and a 
random variable with $L$ possible outcomes:
\begin{align}\label{Discr_Y}
 \Delta Y = Y_{n+1} - Y_n &= (L-l) \log(u) + (l-1) \log(d) \\ \nonumber
  &= \bar b\, \Delta t + (L-2l+1) \alpha(\Delta t).
\end{align}
The term $\bar b\, \Delta t$ is the drift term, while the term $l$ is a random variable that takes values in $\{1,2,...,L\}$ with probability $p_l$. 
It has to satisfy the martingale condition:
$$ \E \bigl[(L-2l+1) \alpha(\Delta t) \bigr] = \alpha(\Delta t) \sum_{l=1}^L p_l (L-2l+1) = 0, $$
and $\alpha(\Delta t)$ is a function of $\Delta t$.

The corresponding up/down factors have the following representation:
\begin{equation}\label{updown}
 u = \exp\biggl( \frac{b}{L-1} + \alpha(\Delta t) \biggr) \hspace{2em}  d = \exp\biggl( \frac{b}{L-1} - \alpha(\Delta t) \biggr),
\end{equation}
and we can readily see that if $u/d$ is a constant, then the tree recombines.

Given the mean $c_1 = \E[\Delta Y] = \bar b \Delta t$, the $k$-central moment is:
\begin{equation}\label{higher_moments}
 \E \bigl[ (\Delta Y - c_1)^k \bigr] = \alpha(\Delta t)^k\, \E \bigl[ (L-2l+1)^{k} \bigr].
\end{equation}
The moment matching condition requires that the central moments of the discrete process (\ref{Discr_Y}) 
are equal to the central moments
of the continuous process (\ref{log_martingale}):
\begin{equation}\label{moment_matching}
 \alpha(\Delta t)^k\, \E \bigl[ (L-2l+1)^{k} \bigr] = \mu_k.
\end{equation}
We fix $L=5$, and using the relation between central moments and cumulants (Eq. (\ref{moment_cumulants}) )
we solve the linear system of equations for the transition probabilities:
\begin{align}\label{probabilities1}
 p_1 &= \frac{1}{196 \alpha(t)^4} \biggl[ \frac{3}{2} c_2^2 -2 c_2\alpha(t)^2 + 2 c_3 \alpha(t) +\frac{1}{2} c_4  \biggr] \\ \nonumber
 p_2 &= \frac{1}{196 \alpha(t)^4} \biggl[ -6 c_2 + 32c_2 \alpha(t)^2 - 4c_3 \alpha(t) -2 c_4 \biggr] \\ \nonumber
 p_3 &=  1 + \frac{1}{196 \alpha(t)^4} \biggl[ 3c_4 + 9c_2^2 -60c_2 \alpha(t)^2   \biggr] \\ \nonumber
 p_4 &= \frac{1}{196 \alpha(t)^4} \biggl[ -6 c_2 + 32c_2 \alpha(t)^2 + 4c_3 \alpha(t) -2 c_4  \biggr] \\ \nonumber
 p_5 &= \frac{1}{196 \alpha(t)^4} \biggl[ \frac{3}{2} c_2^2 -2 c_2\alpha(t)^2 - 2 c_3 \alpha(t) +\frac{1}{2} c_4 \biggr].
\end{align}
The drift parameter corresponds to $\bar b = r - w + \theta$.
The only missing term to find is $\alpha(\Delta t)$. This is a function of the time increment $\Delta t$ and can be determined using the 
higher order terms in the moment matching condition together with the condition of positive probabilities.

Recall that the well known binomial model \cite{CRR79} assumes the value $\alpha(\Delta t) = \sigma \sqrt{\Delta t}$,
that represents the volatility of the increments in the time interval $\Delta t$.
In the trinomial model, the parameter $\alpha(\Delta t)$ assumes the value $\alpha(\Delta t) = \frac{1}{2} \sigma \sqrt{3\Delta t}$, see for instance \cite{YaPr01}.
For the pentanomial method a good representation for $\alpha(\Delta t)$ is:
\begin{equation}\label{alphat}
 \alpha(\Delta t) = \sqrt{c_2} \sqrt{\frac{3+\bar \kappa}{12}},
\end{equation}
where $\bar \kappa = c_4 / c_2^2$ is the excess of kurtosis\footnote{We use the bar over $\kappa$, 
to distinguish the kurtosis from the variance of the gamma process $\kappa$.}. 
We refer to \cite{YaPr06} for the derivation.
This choice guarantees that the probabilities $p_i$ for $i=1...5$ are always positive and sum to one. We can replace the expression
(\ref{alphat}) inside (\ref{probabilities1}), to obtain the simpler form:
\begin{align}\label{probabilities2}
 & [p_1,p_2,p_3,p_4,p_5] \approx \biggl[ \frac{3+\bar \kappa+s\sqrt{9+3\bar \kappa}}{4(3+\bar \kappa)^2} , 
 \frac{3+\bar \kappa-s\sqrt{9+3\bar \kappa}}{2(3+\bar \kappa)^2} , \\ \nonumber
 &
 \frac{3+2\bar \kappa}{2(3+\bar \kappa)} ,
 \frac{3+\bar \kappa+s\sqrt{9+3\bar \kappa}}{2(3+\bar \kappa)^2} ,
 \frac{3+\bar \kappa-s\sqrt{9+3\bar \kappa}}{4(3+\bar \kappa)^2} \biggr],
\end{align}
where $s = c_3 / \sqrt{c_2^3}$ is the skewness.
\begin{Remark}
 The standard deviation of every L\'{e}vy process with finite second moment follows the square root propagation rule. This means that the term $\alpha(\Delta t)$ has to be proportional
 to the square root of $\Delta t$. In the binomial and trinomial models, the proportionality constant is explicit, while for the pentanomial method it is implicit
 in the formula (\ref{alphat}). Expanding the formula using the expression (\ref{VG_cumulants}) for the cumulants, it is possible to check that the square root rule is
 satisfied at first order in $\sqrt{\Delta t}$.
\end{Remark}



\subsection{Convergence}


We call a generic jump process (\ref{Levy_Ito3}) with first four cumulants $c_1$,$c_2$,$c_3$,$c_4$ as in (\ref{VG_cumulants}), 
the \emph{approximated process} $X^A$. 
The cumulant generating function of the increment $\Delta X^A$ has the following series representation (see Section \ref{cumulant_sec} ):
\begin{equation}\label{cum_gen_appr}
 H_{\Delta X^{A}}(u) = ic_1 u -\frac{c_2u^2}{2} -\frac{ic_3u^3}{3!} + \frac{c_4u^4}{4!} + \mathcal{O}(u^5).
\end{equation}
We are interested in the approximation of a VG process with drift (\ref{log_martingale}), therefore we require that $c_1 = \bar b \Delta t = (r-w+\theta)\Delta t$. 
\begin{Theorem}
The increments of the discrete Markov chain (\ref{Discr_Y}) and the increments of the approximated process $X^A$ have the same distribution by construction.
\end{Theorem}
\begin{proof}
The idea of the proof is to show that the cumulant generating function of the discrete process (\ref{Discr_Y}) 
coincides with that of the approximated process (\ref{cum_gen_appr}). We prove it by using the moment
matching condition (\ref{moment_matching}).
\begin{align}
H_{\Delta Y}(u) &= \log \bigl( \phi_{\Delta Y}(u)  \bigr) = \log \biggl( \E \bigl[ e^{iu \Delta Y} \bigr] \biggr) \\ \nonumber
                &= \log \biggl( \E \biggl[ e^{iu \bigl( \bar b \Delta t + (L-2l+1) \alpha(\Delta t) \bigr) } \biggr] \biggr) \\ \nonumber
		&= iu \bar b \Delta t + \log \biggl( \E \biggl[ e^{iu \bigl(  (L-2l+1) \alpha(\Delta t) \bigr) } \biggr] \biggr).
\end{align}
We can expand the exponential function in Taylor series and use the moment matching condition (\ref{moment_matching}) to obtain:
\begin{align}
H_{\Delta Y}(u) &= iu \bar b \Delta t + \log \biggl( \sum_{k=0}^{\infty} \frac{(iu)^k}{k!} 
\bigl(\alpha(\Delta t)\bigr)^k \E \biggl[ \bigl( L-2l+1  \bigr)^k \biggr] \biggr) \\ \nonumber
                &= iu \bar b \Delta t + \log \biggl( \sum_{k=0}^{\infty} \frac{(iu)^k}{k!} \mu_k \biggr) \\ \nonumber
                &= iu c_1 + \sum_{k=2}^{\infty} \frac{(iu)^k}{k!} c_k \\ \nonumber 
                &= H_{\Delta X^{A}}(u),
\end{align}
\end{proof}
\begin{Remark}
All the cumulants of $\Delta X^A$ are equal to the cumulants of the Markov chain (\ref{Discr_Y}) by construction, but only the first four are equal to the VG cumulants.
When all the cumulants $c_i$, for $0 \leq i \leq n$, are equal to the VG cumulants, the approximated process $X^A$ converges in distribution to
the original VG process for $n \to \infty$.
In order to describe $n$ cumulants, we need $n+1$ branches. Therefore, when the number of cumulants of $\Delta X^A$ that are equal to those of the VG goes to infinity, 
the number of branches has to go to infinity as well.
We assume that five branches ($L=5$) are enough to describe the main features of the underlying process and, at the same time, keep the numerical
problem simple. 
\end{Remark}

\begin{Theorem}
The distribution of the pentanomial tree at time $N$ converges to the distribution of a compound Poisson process at time $N$ with $L=5$ possible jump sizes and activity $\lambda = \frac{3}{2 \bar \kappa N}$, when $\Delta t \to 0$.   
\end{Theorem}
For the proof of this theorem  
we refer to Section 4.2 of \cite{YaPr06}. The authors first define the jump sizes and their respective probabilities, and then
prove that when $\Delta t \to 0$ the characteristic function of the pentanomial tree converges to the 
characteristic function of the compound Poisson process.



\section{Numerical results} \label{sec4_ch3}

In this section we present the steps to implement the algorithm for pricing European and American options with the multinomial method.
Then we compare the results with those obtained by the PIDE method and Black-Scholes model.

\subsection{Algorithm}

We suggest the following algorithm for pricing with the multinomial method:
\begin{enumerate}
 \item Compute the transition probabilities vector (\ref{probabilities2}). 
 \item Compute the up/down factors $u$ and $d$ (\ref{updown}) and the vector of prices $S_N$ at terminal time $N$ as in Eq. (\ref{Discr_S2}).
 \item Evaluate the payoff of the option $V^N(S_N)$ at terminal time $N$.
 \item Given the option values at time $t_{n+1}$ compute the values at time $t_n$. The value is the conditional expectation:
 \begin{equation}
 V^n(s^{(k)}_n) = e^{-r\Delta t} \E^{\Q} \biggl[ V^{n+1}(S_{n+1}) \bigg| S^{(k)}_n = s^{(k)}_n \biggr]. 
\end{equation}
 \item If computing the price of an American option, the value at the previous time level is the maximum between the conditional expectation and
 the intrinsic value of the option. For an American put we have:
 \begin{equation}
 V^n(s^{(k)}_n) = \max \biggr \{ e^{-r\Delta t} \E^{\Q} \biggl[ V^{n+1}(S_{n+1}) \bigg| S^{(k)}_n = s^{(k)}_n \biggr] , K-s^{(k)}_n \biggr \}. 
\end{equation}	
 \item Iterate the algorithm until the initial time $t_0$. 
\end{enumerate}

In all the numerical computations of this chapter we consider the risk neutral VG parameters in Table \ref{sample-table}. These parameters correspond to the parameters used 
in \cite{Canta2}.

\begin{table}[!h]
\centering
{\begin{tabular}{llll}
\toprule
 \multicolumn{4}{c}{Parameters} \\
\midrule
$r$ & $\theta$ & $\sigma$ & $\kappa$ \\ 
0.06 & -0.1 & 0.2 & 0.2 \\
\bottomrule
\end{tabular}}
\caption{$r$ is the risk free interest rate and $\theta, \sigma, \kappa$ are the risk neutral VG parameters.}
\label{sample-table}
\end{table}

\begin{Remark}
In practice, the risk neutral parameters of the VG model should be calibrated from market prices of European vanilla options, using the closed pricing formula of \cite{MCC98}.
After that, the risk neutral VG model can be used to price exotic derivatives for which there are no available closed formulas.
The multinomial method is particularly suitable for American-style options, since Monte Carlo and PIDEs methods are in general more difficult to implement. 
For this reason, tree methods are quite popular among practitioners. 
\end{Remark}







\subsection{European options}


We compare the numerical results for European call and put options obtained with the multinomial and the PIDE approaches.
We solve the VG PIDE following the IMEX method introduced in Section \ref{VG_section2}.
The details of the implementation are presented in Section \ref{numerical_concergence_section}. 

Following the algorithm proposed in the previous section, we solve the option pricing problem using the multinomial method. 
The number of time steps for all the computations presented in the following pictures is $N=2000$. 
The Table \ref{Convergence} 
shows a convergence analysis for the prices of European calls, puts and American puts. We can see that the convergence is quite fast.  
\begin{table}[ht]
\centering
{\begin{tabular}{llllll}
\toprule
  $N$ & Eu. Call & Eu. Put & Time & Am. Put & Time \\
\midrule
    50 & 4.41873125 & 2.08928091 & 0.001 & 2.36765911 & 0.007 \\
    100 & 4.41960265 & 2.09015381 & 0.002 & 2.37255454 & 0.02 \\
    200 & 4.41997010 & 2.09052201 & 0.004 & 2.37480218 & 0.07 \\
    400 & 4.42013640 & 2.09068869 & 0.01 & 2.37587117 & 0.29 \\
    800 & 4.42021515 & 2.09076762 & 0.03 & 2.37639131 & 1.09 \\
    1000 & 4.42023054 & 2.09078306 & 0.04 & 2.37649417 & 1.67 \\
    1500 & 4.42025089 & 2.09080345 & 0.06 & 2.37663070 & 3.79 \\
    2000 & 4.42026098 & 2.09081357 & 0.10 & 2.37669869 & 6.80 \\
    2500 & 4.42026701 & 2.09081962 & 0.16 & 2.37673941 & 10.65 \\
    3000 & 4.42027102 & 2.09082364 & 0.2 & 2.37676652 & 14.78 \\
  \bottomrule
\end{tabular}}
\caption{Convergence table for ATM European and American options with strike $K=40$ and $T=1$. The time unit is in seconds.}
\label{Convergence}
\end{table}
Figures \ref{figCall} and \ref{figPut} show the single prices obtained by the multinomial method compared with the curve obtained by solving the PIDE.
In Table \ref{Option_values} we compare directly the call/put numerical values obtained with the two methods.
\begin{figure}[t!]
 \begin{minipage}[b]{0.5\linewidth}
   \centering
 \includegraphics[width=\linewidth]{EU_call.png}
 % EU_call.png: 0x0 pixel, 300dpi, 0.00x0.00 cm, bb=
 \caption{European call option with strike $K=40$ and time to maturity 1 year.}
 \label{figCall}
  \end{minipage}
 \ \hspace{2mm} \hspace{3mm} \
 \begin{minipage}[b]{0.5\linewidth}
 \centering
 \includegraphics[width=\linewidth]{EU_put.png}
 % EU_call.png: 0x0 pixel, 300dpi, 0.00x0.00 cm, bb=
 \caption{European put option with strike $K=40$ and time to maturity 1 year.}
 \label{figPut}
 \end{minipage}
\end{figure}

\begin{table}[ht]
\centering
{\begin{tabular}{lllll} 
\toprule
\multicolumn{5}{c}{Different methods} \\
\midrule
$S_0$ & PIDE Call & Multi Call & PIDE Put & Multi Put \\
\midrule
36 & 2.1036 & 2.1131 & 3.7842 & 3.7837  \\
  38 & 3.1163 & 3.1051 & 2.7893 & 2.7756 \\
  40 & 4.4162 & 4.4203 & 2.0852 & 2.0908 \\
  42 & 5.8335 & 5.8309 & 1.5050 & 1.5014 \\
  44 & 7.4417 & 7.4524 & 1.1132 & 1.1229 \\
\bottomrule
\end{tabular}}
\caption{European Options, with strike $K=40$ and $T=1$.}
\label{Option_values}
\end{table}



\subsection{American options}

In this section we present the numerical results obtained with the multinomial method algorithm for American put options, and compare them with the PIDE method (see fig. \ref{AmVG}).
The PIDE (\ref{VG_JD}) is modified in order to take into account the early exercise feature:
\begin{align}\label{VG_Am_JD}
&  \min \biggl\{ - \frac{\partial V(t,x)}{\partial t} -
 \bigl( r-\frac{1}{2}\sigma_{\epsilon}^2 - w_{\epsilon} \bigr) \frac{\partial V(t,x)}{\partial x} 
 - \frac{1}{2}\sigma_{\epsilon}^2 \frac{\partial^2 V(t,x)}{\partial x^2} + (\lambda_{\epsilon} + r) V(t,x) \\ \nonumber
 &- \int_{|z| \geq \epsilon} V(t,x+z) \nu(dz) \, , \, \biggl( V(t,x) - (K-e^x)^+ \biggr) \biggr\} = 0.
\end{align}
To solve this equation we use the same discretization and same settings used in Sections \ref{VG_section2} and \ref{numerical_concergence_section} for the European option problem.

The numerical values obtained by multinomial and PIDE methods are collected in Tab. \ref{Option_values3}.
The run times for the multinomial algorithm are shown in the convergence Table \ref{Convergence}.
The run times in the last column of Table \ref{Convergence} (for the American put option) can be compared with run times of the PIDE method, reported in Table \ref{PIDE_times}. 
We can see that the difference is quite big. In order to achieve a precision of three decimal digits the PIDE algorithm takes more than two minutes, while the multinomial algorithm
can achieve a precision of four decimal digits in about 10 seconds. 
On the other hand, the multinomial prices are only approximations because are calculated considering the approximated process $X^A$ in place of the original VG process. 
This is the reason of the discrepancy in the prices presented in the two tables, which still is a discrepancy of just $1\%$.
\begin{table}[ht]
\centering
{\begin{tabular}{llll} 
\toprule
 M & N & PIDE Put & Time (seconds) \\
\midrule
 8000  & 6000  & 2.3493 & 10 \\
 16000 & 12000 & 2.3521 & 50 \\
 25000 & 20000 & 2.3530 & 143 \\
 30000 & 25000 & 2.3532 & 229 \\
\bottomrule
\end{tabular}}
\caption{American put option, with strike $K=40$ and $T=1$ and parameters in Tab. \ref{sample-table}. Computational times for $M$ space steps and $N$ time steps.}
\label{PIDE_times}
\end{table}

\begin{figure}[ht!]
 \centering
 \includegraphics[scale=0.4]{American_VG.png}
 % EU_call.png: 0x0 pixel, 300dpi, 0.00x0.00 cm, bb=
 \caption{American put option with strike $K=40$ and time to maturity 1 year. Comparison of PIDE prices and multinomial prices.}
 \label{AmVG}
\end{figure}
\begin{table}[ht]
{\begin{tabular}{llllll}
\toprule
 \multicolumn{6}{c}{Prices comparison} \\
\midrule
$S_0$ & BS Eu. Put & VG Eu. Put & BS Am. Put & VG Am. Put & VG Am. PIDE Put \\
 \midrule
  30 & 8.1316 & 8.0809 & 10     & 10     & 10 \\
  32 & 6.5292 & 6.4055 & 8      & 8      & 8 \\
  34 & 5.1169 & 4.9851 & 6.0894 & 6      & 6 \\
  36 & 3.9150 & 3.7837 & 4.5415 & 4.3173 & 4.3982 \\
  38 & 2.9263 & 2.7756 & 3.3264 & 3.2034 & 3.2195 \\
  40 & 2.1388 & 2.0908 & 2.3924 & 2.3767 & 2.3521 \\
  42 & 1.5322 & 1.5014 & 1.6911 & 1.6947 & 1.6849 \\
  44 & 1.0766 & 1.1229 & 1.1755 & 1.2267 & 1.2118 \\
  46 & 0.7433 & 0.7858 & 0.8043 & 0.8699 & 0.8650 \\
  48 & 0.5049 & 0.5787 & 0.5425 & 0.6310 & 0.6221 \\
  50 & 0.3384 & 0.4259 & 0.3612 & 0.4661 & 0.4480 \\ 
  52 & 0.2238 & 0.3015 & 0.2376 & 0.3242 & 0.3222 \\
  55 & 0.1178 & 0.1909 & 0.1243 & 0.2051 & 0.1990 \\
  60 & 0.0386 & 0.0880 & 0.0404 & 0.0942 & 0.0913 \\ 
 \bottomrule
 \end{tabular}}
  \caption{Values for European and American put options using Black-Scholes (binomial) and Variance Gamma (multinomial) methods. 
  Strike $K=40$ and $T=1$. The BS volatility have same value of the VG volatility  
  $ \sigma^{BS} = (\sigma^2 + \theta^2 \kappa) = 0.2049$. 
  The last column presents values of American options computed with the PIDE method using a grid with $M=16000$ and $N=12000$ steps. }
 \label{Option_values3}
\end{table}

In Table \ref{Option_values3}, we consider also European and American put option prices 
calculated with the Black-Scholes (BS) models.
The BS volatility is chosen equal to the VG volatility $ \sigma^{BS} = (\sigma^2 + \theta^2 \kappa)$.
As expected, the deep OTM (out of the money) prices computed under VG are higher than the corresponding prices computed under BS. This is a consequence of the
shape of the VG density function (\ref{VG_density}), which has \emph{heavier tails} than the normal distribution. 
This means that the probability of a deep OTM option to return in the money, 
is higher if calculated with the VG model than BS, and therefore we get higher option prices.

The Black-Scholes prices are computed using a binomial algorithm.  
The same values can be obtained using the multinomial algorithm for the VG process and setting $\theta = \kappa = 0$ and $\sigma = \sigma^{BS}$.
Recall that under the Black-Scholes model, the log-returns follow a Brownian motion. 
Looking at the definition of the VG process (\ref{VG_process}), it is evident that when $\theta$ and $\kappa$ are zero, the process becomes a Brownian motion:
$$ X^{VG}_t \underset{\theta,\kappa \to 0}{\to} \sigma W_t. $$
It follows that the price process converges to the Geometric Brownian Motion:
$$ S_t = S_0 e^{(r-w)t + X_t} \underset{\theta,\kappa \to 0}{\to} S_0 e^{(r-\frac{1}{2}\sigma^2)t + \sigma W_t} $$
where:
\begin{align*}
 \lim_{\theta,\kappa \to 0} w &= \lim_{\theta,\kappa \to 0} -\frac{1}{\kappa} \log(1-\theta \kappa -\frac{1}{2}\sigma^2 \kappa) \\
 &= -\frac{1}{2}\sigma^2.
\end{align*}




\section{Chapter conclusions}


In this chapter we make a small digression from the main theme of the thesis by presenting the application of a multinomial approximation for option pricing 
using the Variance Gamma model.
The main results of this chapter are published in the paper \cite{Canta2}. 

The VG process is approximated by a general jump process that has the same first four cumulants of the original VG process. 
We show that by construction the multinomial method has the same distribution of the approximated process. 
We provide numerical results for European and American options and compare them with
results obtained by PIDE methods, and with Black–Scholes prices.

Pricing vanilla call and put European options under the VG model is quite simple, and the best approach is to use the closed formula derived in \cite{MCC98}.
The advantage of the multinomial method is in the computation of American options prices, where there is no closed formula and all the other approaches, 
such as PIDEs and Least Squares Monte Carlo (proposed in \cite{LoSc01} for diffusion processes), are difficult to implement. \\
It turns out that the
multinomial method is straightforward to implement. The algorithm does
not involve any matrix multiplications, matrix inversions or decompositions as for PIDEs. Moreover, the IMEX scheme requires also the computation of an integral such as 
(\ref{trap_quad}), which in general can be computationally expensive. 

In order to show the ease of use of this algorithm, in Chapter \ref{Chapter6} we applied it to solve the option pricing problem with 
transaction costs, with good performances.
